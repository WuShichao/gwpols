\pdfoutput=1
%% ****** Start of file apstemplate.tex ****** %
%%
%%
%% This file is part of the APS files in the REVTeX 4 distribution.
%% Version 4.1r of REVTeX, August 2010
%%
%%
%% Copyright (c) 2001, 2009, 2010 The American Physical Society.
%%
%% See the REVTeX 4 README file for restrictions and more information.
%%
%
% This is a template for producing manuscripts for use with REVTEX 4.0
% Copy this file to another name and then work on that file.
% That way, you always have this original template file to use.
%
% Group addresses by affiliation; use superscriptaddress for long
% author lists, or if there are many overlapping affiliations.
% For Phys. Rev. appearance, change preprint to twocolumn.
% Choose pra, prb, prc, prd, pre, prl, prstab, prstper, or rmp for journal
% Add 'draft' option to mark overfull boxes with black boxes
% Add 'showpacs' option to make PACS codes appear
% Add 'showkeys' option to make keywords appear
\documentclass[aps,prd,twocolumn,superscriptaddress,preprintnumbers,floatfix,nofootinbib]{revtex4-2}
%\documentclass[aps,prl,preprint,superscriptaddress]{revtex4-1}
%\documentclass[aps,prl,reprint,groupedaddress]{revtex4-1}

\usepackage{graphicx}
\usepackage{amsmath}
%\usepackage{mdwlist}
\usepackage[caption=false]{subfig}
\usepackage{siunitx}
\usepackage{placeins}
\usepackage{color}
\usepackage{standalone}
\usepackage{dcolumn}
\usepackage{tensor}
\usepackage{bm}
%\usepackage{MnSymbol}
\usepackage{microtype}
\usepackage{etoolbox}
\usepackage{amssymb}
\usepackage{mathrsfs}
\usepackage{accents}
\usepackage[normalem]{ulem}
\usepackage[dvipsnames]{xcolor}
\usepackage[colorlinks,urlcolor=NavyBlue,citecolor=NavyBlue,linkcolor=NavyBlue,pdfusetitle]{hyperref}
\usepackage[all]{hypcap}
\usepackage[inline]{enumitem}
\usepackage[utf8]{inputenc}
\usepackage{csquotes}
\usepackage{array}
\usepackage{booktabs}

\newcommand{\ts}{\textsuperscript}

\newcommand{\beq}{\begin{equation}}
\newcommand{\eeq}{\end{equation}}

\newcommand{\nn}{\nonumber}

\newcommand*{\eq}[1]{Eq.~\eqref{eq:#1}}
\newcommand*{\fig}[1]{Fig.~\ref{fig:#1}}
\newcommand*{\sect}[1]{Sec.~\ref{sec:#1}}

\newcommand{\boldmu}{\boldsymbol{\mu}}
\newcommand{\boldn}{\mathbf{n}}
\newcommand{\boldd}{\mathbf{d}}
\newcommand{\bolds}{\mathbf{s}}
\newcommand{\boldSigma}{\boldsymbol{\Sigma}}

\newcommand{\area}{\mathcal{A}}

%% Try to control orphans, widows, and extra whitespace
%\widowpenalty=10000
%\clubpenalty=10000
%\raggedbottom
\interfootnotelinepenalty=3000

%% Enable/disable comments
\newtoggle{commentsoff}
\togglefalse{commentsoff}
%\toggletrue{commentsoff}

% Check for command-line option to turn comments off
% https://tex.stackexchange.com/questions/1492/passing-parameters-to-a-document
\ifdefined\nocomments
    \toggletrue{commentsoff}
\fi

\iftoggle{commentsoff}{
  \newcommand*{\mi}[1]{}
  \newcommand*{\mg}[1]{}
  \newcommand*{\wf}[1]{}
  \newcommand*{\comment}[1]{}
  \newcommand*{\suggest}[3]{#2}
  \newcommand*{\todo}[1]{}
  \newcommand*{\warn}[1]{}
  \newcommand*{\red}[1]{#1}
  \newcommand*{\blue}[1]{#1}
  \newcommand{\cn}{}
  \newcommand*{\commentmark}[1]{}
}{
  \newcommand*{\mi}[1]{{\color{magenta} [{\bf MAX}: #1]}}
  \newcommand*{\wf}[1]{\textcolor{green}{\textbf{WILL}: #1}}
  \newcommand*{\suggest}[3]{\textcolor{Purple}{\textbf{#3} \sout{#1} #2}}
  \newcommand*{\comment}[1]{{\color{blue} [{\bf NOTE}: #1]}}
  \newcommand*{\warn}[1]{{\color{red} [{\bf WARNING}: #1]}}
  \newcommand*{\todo}[1]{{\color{red} [{\bf TODO}: #1]}}
  \newcommand*{\red}[1]{{\color{purple} #1}}
  \newcommand*{\blue}[1]{{\color{blue} #1}}
  \newcommand{\cn}{\blue{\bf [cn]}}
}

\graphicspath{{./fig/}}

\newcommand{\dcc}{LIGO-PXXXXXXX}

% NOTATION MACROS
\newcommand{\infd}{\mathrm{d}}
\newcommand{\white}{\bar}

\newcommand{\snropt}{\mathrm{SNR}}
\newcommand{\snrmf}{\mathrm{SNR}_{\rm mf}}
\newcommand{\cov}{C}
\newcommand{\acf}{\rho}
\newcommand{\nmode}{D}

\begin{document}

% Use the \preprint command to place your local institutional report
% number in the upper righthand corner of the title page in preprint mode.
% Multiple \preprint commands are allowed.
% Use the 'preprintnumbers' class option to override journal defaults
% to display numbers if necessary
% \preprint{\dcc}

%Title of paper
\title{Parametrizing gravitational-wave polarizations}

\author{Maximiliano Isi}
\email[]{maxisi@mit.edu}
\thanks{NHFP Einstein fellow}
%\homepage[]{Your web page}
%\thanks{}
%\altaffiliation{}
\affiliation{
LIGO Laboratory, Massachusetts Institute of Technology, Cambridge, Massachusetts 02139, USA
}%
%\affiliation{Center for Computational Astrophysics, Flatiron Institute, 162 5th Ave, New York, NY 10010}

% Because hyperref only gets the *last* author, we need to be explicit.
\hypersetup{pdfauthor={Isi}}

\date{\today}

\begin{abstract}
Amazing!
\end{abstract}

\maketitle

% \tableofcontents

\section{Introduction}
\label{sec:intro}

It is useful to parametrize gravitational-wave (GW) polarizations in different ways.
Here I outline some useful alternatives and show how they are related.
It is important to understand the priors implied by the choice of parametrization through the corresponding Jacobians.

\section{General relativity}

\subsection{Polarizations}

In GR, there exist two propagating gravitational degrees of freedom, corresponding to two independent GW polarizations.
Their local effect can be encoded in a driving matrix $h_{ij}$ corresponding to the transverse-traceless part of the metric perturbation.
In a frame with $z$-axis along the direction of propagation, we can write this as
\beq \label{eq:hij}
(h_{ij}) = \begin{pmatrix}
h_+ & h_\times  & 0 \\
h_\times  & - h_+ & 0  \\
0 & 0 & 0
\end{pmatrix} ,
\eeq
where the plus ($+$) and cross ($\times$) polarizations are implicit functions of the retarded time, $t - R/c$, to be specified by the source dynamics and the distance $R$ to the source.
It can be useful to rewrite \eq{hij} as $h_{ij} = h_+ e^+_{ij} + h_\times e^\times_{ij}$, with respect to the polarization tensors
\beq
e^+_{ij} \equiv \hat{x}_i \hat{x}_j - \hat{x}_i \hat{x}_j \, ,
\eeq
\beq
e^\times_{ij} \equiv \hat{x}_i \hat{y}_j + \hat{x}_i \hat{y}_j\, ,
\eeq
where $\hat{x}$ and $\hat{y}$ are orthonormal vectors that, with $\hat{z}$ form a right-handed Cartesian basis; this is the \emph{wave frame}.

Equation \eqref{eq:hij} presumes a specific choice of frame orientation that defines the basis in which the $h_{ij}$ components are written.
Although $\hat{k}$ must be parallel to the (spatial) wave vector $\vec{k}$ for \eq{hij} to hold, there is no a priori restriction on the orientation of the $x$ and $y$ axes within the plane perpendicular to $\vec{k}$.
This is usually specified by an arbitrary \emph{polarization angle} $\psi$, defined with respect to some convenient reference (e.g., the celestial equator).
With some trigonometry, it is straightforward to show that a rotation by some angle $\Delta \psi$ around $z$ leaves the form of \eq{hij} unchanged after redefining
\beq \label{eq:htransf}
h_+ \rightarrow h_+' = h_+ \cos 2\Delta \psi - h_\times \sin 2\Delta\psi \, ,
\eeq
\beq
h_\times \rightarrow h_\times' = h_\times \cos 2\Delta \psi + h_+ \sin 2\Delta\psi \, .
\eeq
Therefore, $h_+$ and $h_\times$ are nothing but the two components of a tensor field with spin-weight $s=2$, and the two polarizations are only defined up to an arbitrary choice of $\psi$.
In fact, we coud easily rewrite \eq{hij} in terms of the corresponding right and left handed polarizations, $h_{R/L} = h_+ \pm i h_\times$, which are invariant under $\psi$ rotations (concretely, eigenvectors of the helicity operator with eigenvalues $\pm 2$).
%Although a choice of $\psi$ is required to prescribe $h_+$ and $h_\times$ in \eq{hij}, the form of this equation is invariant under changes in $\psi$.

In the small-antenna limit, the signal induced by a passing GW on a given detector can be written as the projection
\beq \label{eq:h}
h(t) \equiv D^{ij} h_{ij} = F_+ h_+ + F_\times h_\times\, 
\eeq
with antenna patterns $F_{+/\times} \equiv D^{ij} e^{+/\times}_{ij}$ defined in terms of a detector tensor $D_{ij}$ representing the geometry of the measurement.
For a differential-arm detector with arms pointing along unit vectors $\hat{X}$ and $\hat{Y}$, this is just $D_{ij} = (\hat{X}_i \hat{X}_j - \hat{Y}_i \hat{Y}_j)/2$.%
\footnote{These expressions are valid in the local Lorentz frame of the detector, so we can raise and lower indices with the flat metric.}
In this limit, the antenna patterns are thus purely geometric factors that encode the relative orientations of the detector and wave frames, as defined by $\{\hat{X},\, \hat{Y},\, \hat{Z}\}$ and $\{\hat{x},\, \hat{y},\, \hat{z}\}$ respectively.
Under rotations of the wave frame, they transform through an expression complementary to \eq{htransf},
\beq \label{eq:Ftransf}
h_+ \rightarrow F_+' = F_+ \cos 2\Delta \psi + F_\times \sin 2\Delta\psi \, ,
\eeq
\beq
F_\times \rightarrow F_\times' = F_\times \cos 2\Delta \psi - F_+ \sin 2\Delta\psi \, ,
\eeq
ensuring that the observable $h(t)$ in \eq{h} is independent of the arbitrary angle $\psi$.

\subsection{Templated analyses}

In modeling GW waveforms, it is useful to tie the polarization frame to the geometry of the source.
This is because, although we are free to set our coordinates arbitrarily, we need to make a choice in order to write out explicit expressions for $h_+$ and $h_\times$.
Standardizing this choice reduces ambiguity, and can simplify expressions.

For a compact binary, it is conventional to compute the polarizations in a frame such that $\hat{x}$ lies along the intersection of the orbital plane with the plane of the sky (the line of nodes).
If the spins of the objects are a aligned and the source does not precess, one can use the Einstein field equation to show that, in that frame, $h_+$ and $h_\times$ take the form
\beq \label{eq:cbc_p}
h_+ = \left(1 + \cos^2 \iota\right) A(t; R, \vec{\lambda}) \cos \Phi(t; \vec{\lambda}) \, , 
\eeq
\beq \label{eq:cbc_c}
h_\times = 2 \cos \iota\, A(t; R, \vec{\lambda}) \sin \Phi(t; \vec{\lambda})\, ,
\eeq
where the inclination $\iota$ is the angle between the orbital angular momentum and the line of sight, and $A(t; \vec{\lambda}) \equiv R  A(t; R, \vec{\lambda})$ and $\Phi(t; \vec{\lambda})$ are some amplitude and phasing functions fully determined by intrinsic source parameters $\vec{\lambda}$, in this case encoding the characteristic chirping signal.
The set of parameters $\vec{\lambda}$ include the component masses and spin vectors, a reference phase and, potentially, parameters related to eccentricity and matter effects (like tides).
% The inclination $\iota$ is the polar angle with respect to the orbital angular momentum.

Equations \eqref{eq:cbc_p} and \eqref{eq:cbc_c} describe an elliptically polarized wave.
This is a consequence of the geometry of our assumed source, which is symmetric under reflections around the orbital plane.
The ellipticity is simply the ratio of plus and cross amplitudes in this frame \mi{check sign}
\beq
\epsilon = \frac{2 \cos\iota}{1 + \cos^2\iota}\, .
\eeq
This reveals that our decision to tie the polarization frame to the line of nodes was a good one: because this preserves the symmetries of the source, it also happens to be aligned with the principal directions of the polarization ellipse.

Having specified $h_+$ and $h_\times$ in this standard source-based frame, we need to determine how that frame itself is oriented with respect to the detectors in order to evaluate \eq{h}.
This is most easily done by expressing the components of $h_{ij}$ and $D_{ij}$ in a common reference frame.
In celestial coordinates, which are suitable for networks of GW detectors on Earth, we can \emph{define} $\psi = 0$ to mean that the line of nodes lies along the celestial North.
We may then obtain expressions $h_+$ and $h_\times$ for any other $\psi$ via \eq{htransf}.
In this convention, $\psi$ is identical to the source \emph{position angle}, so we can think of it as a property of the source, rather than an arbitrary parameter orienting our frame.
However, it is useful to keep in mind that this is a conceptual shortcut.

\mi{identify 3 different angles: $\psi$, $\Psi$, $\theta$}

\subsection{Untemplated analyses}

In certain analyses, it is not possible or useful to explicitly tie the polarization frame to properties of the source.
This is usually the case in unmodeled analyses, which are note tailored to any specific source.
\mi{also true for ringdowns, if we don't have a model for the amplitudes}
In that case, the model for $h_+$ and $h_\times$ can be defined in any arbitrary frame.

\section{Nontensor polarizations}

\newcommand{\hx}{h_{\rm x}}
\newcommand{\hy}{h_{\rm y}}
\newcommand{\hb}{h_{\rm b}}
\newcommand{\hlon}{h_{\rm l}}

In GR, there exist two propagating degrees of freedom, corresponding to two independent GW polarizations.
Their local effect can be encoded in a spatial driving matrix $h_{ij}$ that, in a frame with $z$-axis along the direction of propagation, we can write as
\beq \label{eq:polarizations}
(h_{ij}) = \begin{pmatrix}
\hb + h_+ & h_\times  & \hx  \\
h_\times  & \hb - h_+ & \hy  \\
\hx    & \hy    & \hlon
\end{pmatrix} ,
\eeq
with $h_p$, for $p$ in $\{+,\, \times,\, {\rm x},\, {\rm y},\, {\rm b},\, {\rm l}\}$, unspecified functions of time representing the six possible polarization degrees of freedom.
In GR, only the two tensor polarizations, plus ($+$) and cross ($\times$), are allowed, so that $\hx=\hy=\hb=\hlon=0$.
Different theories may also allow vector modes, x and y (also known as $v_1$ and $v_2$), or scalar modes, breathing (b) and longitudinal (l).

Equation \eqref{eq:polarizations} presumes a specific choice of frame orientation.
For $h_{ij}$ to take this simple form, it is necessary for the $z$ axis to be parallel to the (spatial) wave vector $\vec{k}$.
However, there is no restriction regarding the orientation of the $x$ and $y$ axes within the plane perpendicular to $\vec{k}$; this is usually specified by an arbitrary \emph{polarization angle} $\psi$, defined with respect to some convenient reference (e.g., the celestial equator).
The tensor, vector and scalar components of $h_{ij}$ are constructed to be invariant under $\psi$ rotations, with spin-weights $s=2$, $s=1$ and $s=0$ respectively.
This means that, after redefinitions
\beq
h_+ \rightarrow h_+' = h_+ \cos 2\Delta \psi - h_\times \sin 2\Delta\psi 
\eeq
\beq
h_\times \rightarrow h_\times' = h_\times \cos 2\Delta \psi + h_+ \sin 2\Delta\psi 
\eeq
\beq
\hx \rightarrow \hy' = \hx \cos \Delta \psi - \hy \sin \Delta\psi 
\eeq
\beq
\hy \rightarrow \hx' = \hy \cos \Delta \psi + \hx \sin \Delta\psi 
\eeq

\section{Conclusion}

\begin{acknowledgments}
% NASA
M.I.\ is supported by NASA through the NASA Hubble Fellowship
grant No.\ HST-HF2-51410.001-A awarded by the Space Telescope
Science Institute, which is operated by the Association of Universities
for Research in Astronomy, Inc., for NASA, under contract NAS5-26555.
% % LIGO
% M.I.\ is a member of the LIGO Laboratory.
% LIGO was constructed by the California Institute of Technology and
% Massachusetts Institute of Technology with funding from the National
% Science Foundation and operates under cooperative agreement PHY-0757058.
% DCC
This paper carries LIGO document number \dcc{}.
\end{acknowledgments}

\bibliography{refs}

\end{document}
%
% ****** End of file apstemplate.tex ******
